% -- | New Commands | -----------------------------------------------------
\newcommand {\red}[1]    {\textcolor{red}{#1}}
\newcommand {\green}[1]  {\textcolor{green}{#1}}
\newcommand {\blue}[1]   {\textcolor{blue}{#1}}
\newcommand {\dgreen}[1]    {\textcolor{darkgreen}{#1}}
\newcommand {\orange}[1]   {\textcolor{orange}{#1}}

\newcommand{\deutsch}[1]{\foreignlanguage{german}{#1}}

% -- | Comments | -----------------------------------------------------
\newcommand{\CZ}[1]{\blue{{\textbf{CZ:} #1}}}
\newcommand{\PA}[1]{\red{{\textbf{PA:} #1}}}
\newcommand{\JN}[1]{\red{{\textbf{JN:} #1}}}
\newcommand{\MF}[1]{\red{{\textbf{JN:} #1}}}
\newcommand{\ToDo}      {\textcolor{blue}{\footnotesize \textsc{ToDo}}}
\newcommand{\TBC}       {\textcolor{red}{\footnotesize \textsc{TBC}}}


% -- | unit command | -----------------------------------------------------
%  QUELLE: Axel Reichert <reich@mpie-duesseldorf.mpg.de>
%  SYNTAX:
%       \unit{m}
%       \unit[3]{m}             in der eckigen Klammer steht der Wert, 
%       \unitfrac{mJ}{m\,K}     in der geschweiften die Einheit
%       \unitfrac[3]{mJ}{m\cdot K}
\DeclareRobustCommand{\unit}[2][]{%                
        \begingroup%
                \def\0{#1}%
                \expandafter%
        \endgroup%
        \ifx\0\@empty%
                \ensuremath{\mathrm{#2}}%
        \else%
                \ensuremath{#1\,\mathrm{#2}}%
        \fi%
        }
\DeclareRobustCommand{\unitfrac}[3][]{%
        \begingroup%
                \def\0{#1}%
                \expandafter%
        \endgroup%
        \ifx\0\@empty%
                \raisebox{0.98ex}{\ensuremath{\mathrm{\scriptstyle#2}}}%
                \nobreak\hspace{-0.15em}\ensuremath{/}\nobreak\hspace{-0.12em}%
                \raisebox{-0.58ex}{\ensuremath{\mathrm{\scriptstyle#3}}}%
        \else
                \ensuremath{#1}\,%
                \raisebox{0.98ex}{\ensuremath{\mathrm{\scriptstyle#2}}}%
                \nobreak\hspace{-0.15em}\ensuremath{/}\nobreak\hspace{-0.12em}%
                \raisebox{-0.58ex}{\ensuremath{\mathrm{\scriptstyle#3}}}%
        \fi%
}

%
% --| Abbrevitations |----------------------------------------------------
%
\newcommand{\ie}{i.\,e.\;}%   % --> i.e.
\newcommand{\eg}{e.\,g.\;}%   % --> e.g.

\newcommand{\run}[1]{\textsc{Run\,#1}}


%
% -- | variables | --------------------------------------------------------
%

\newcommand{\dNdeta}{\ensuremath{\mathrm{d}N_{\rm ch}/\mathrm{d}\eta}\xspace}

%
% -- | pictures | ---------------------------------------------------------
%
\newlength{\smallerpicsize}
\setlength{\smallerpicsize}{70mm}
\newlength{\smallpicsize}
\setlength{\smallpicsize}{90mm}
\newlength{\mediumpicsize}
\setlength{\mediumpicsize}{120mm}
\newlength{\largepicsize}
\setlength{\largepicsize}{150mm}

% short caption for the TOC, then normal caption
\newcommand{\PICX}[5]{
   \begin{figure}[!hbt]
      \begin{center}
         \vspace{3ex}
         \includegraphics[width=#3]{#1}
         \caption[#4]{\label{#2} #5}        
      \end{center}  
   \end{figure}
}


% short caption for the TOC, then normal caption
% at the "here"
\newcommand{\PICH}[5]{
   \begin{figure}[H]
      \begin{center}
         \vspace{3ex}
         \includegraphics[width=#3]{#1}
         \caption[#4]{\label{#2} #5}        
      \end{center}  
   \end{figure}
}

%
% -- | references | -------------------------------------------------------
%
% Use as references to figures, tables, etc. 
%  -> use capital ones for beginning of the sentences
%
\newcommand{\figs}{Figs.\xspace}
\newcommand{\Figs}{Figures\xspace}
\newcommand{\eqn}{equation\xspace}
\newcommand{\Eqn}{Equation\xspace}
%
\newcommand{\figref}[1]{Fig.~\ref{#1}}
\newcommand{\Figref}[1]{Figure~\ref{#1}}
%
\newcommand{\tabref}[1]{Tab.~\ref{#1}}
\newcommand{\Tabref}[1]{Table~\ref{#1}}
%
\newcommand{\appref}[1]{appendix~\ref{#1}}
\newcommand{\Appref}[1]{Appendix~\ref{#1}}
%
\newcommand{\secs}{Secs.\xspace}
\newcommand{\Secs}{Sections\xspace}
\newcommand{\secref}[1]{Sec.~\ref{#1}}
\newcommand{\Secref}[1]{Section~\ref{#1}}
%
\newcommand{\chaps}{Chaps.\xspace}
\newcommand{\Chaps}{Chapters\xspace}
\newcommand{\chapref}[1]{Chap.~\ref{#1}}
\newcommand{\Chapref}[1]{Chapter~\ref{#1}}
%
\newcommand{\lstref}[1]{Listing~\ref{#1}}
\newcommand{\Lstref}[1]{Listing~\ref{#1}}
%
% -- | Tables | ------------------------------------------------------------
%
\newcommand{\otoprule}{\midrule[\heavyrulewidth]}
\topfigrule
%
% -- | Analysis commands | -------------------------------------------------
\newcommand {\stat}     {({\it stat.})~}
\newcommand {\syst}     {({\it syst.})~}
 \newcommand {\mom}       {\ensuremath{p}}
\newcommand {\pT}        {\pt}
\newcommand {\meanpT}    {\ensuremath{\langle p_{\mathrm{T}} \kern-0.1em\rangle}\xspace}
\newcommand {\mean}[1]   {\ensuremath{\langle #1 \kern-0.1em\rangle}\xspace} 
\newcommand {\sqrtsNN}   {\ensuremath{\sqrt{s_{\textsc{NN}}}}\xspace}
\newcommand {\sqrts}     {\ensuremath{\sqrt{s}}\xspace}
\newcommand {\vf}        {\ensuremath{v_{\mathrm{2}}}\xspace}
\newcommand {\et}        {\ensuremath{E_{\mathrm{t}}}\xspace}
\newcommand {\mT}        {\ensuremath{m_{\mathrm{T}}}\xspace}
\newcommand {\mTmZero}   {\ensuremath{m_{\mathrm{T}} - m_0}\xspace}
\newcommand {\minv}      {\mbox{$m_{\ee}$}}
%\newcommand {\ee}        {\mbox{e$^+$e$^-$}}
\newcommand {\rap}       {\mbox{$y$}}
\newcommand {\absrap}    {\mbox{$\left | y \right | $}}
\newcommand {\rapXi}     {\mbox{$\left | y(\rmXi) \right | $}}
\newcommand {\abspseudorap} {\mbox{$\left | \eta \right | $}}
\newcommand {\pseudorap} {\mbox{$\eta$}}
\newcommand {\cTau}      {\ensuremath{c\tau}}
\newcommand {\sigee}     {$\sigma_E$/$E$}
\newcommand {\dNdy}      {\ensuremath{\mathrm{d}N/\mathrm{d}y}}
\newcommand {\dNdpt}     {\ensuremath{\mathrm{d}N/\mathrm{d}\pT }}
\newcommand {\dNdptdy}   {\ensuremath{\mathrm{d^{2}}N/\mathrm{d}\pT\mathrm{d}y }}
\newcommand {\fracdNdptdy}   {\ensuremath{ \frac{\mathrm{d^{2}}N}{\mathrm{d}\pT\mathrm{d}y } }}
\newcommand {\dNdmtdy}   {\ensuremath{\mathrm{d^{2}}N/\mathrm{d}\mT\mathrm{d}y }}
\newcommand {\dN}        {\ensuremath{\mathrm{d}N }}
\newcommand {\dNsquared} {\ensuremath{\mathrm{d^{2}}N }}
\newcommand {\dpt}       {\ensuremath{\mathrm{d}\pT }}
\newcommand {\dy}        {\ensuremath{\mathrm{d}y}}
\newcommand {\dNdyBold}  {\ensuremath{\boldsymbol{\dN/\dy}}\xspace}
\newcommand {\dNchdy}    {\ensuremath{\mathrm{d}N_\mathrm{ch}/\mathrm{d}y }\xspace}
\newcommand {\dNchdeta}  {\ensuremath{\mathrm{d}N_\mathrm{ch}/\mathrm{d}\eta }\xspace}
\newcommand {\dNchdptdeta}  {\ensuremath{\mathrm{d}N_\mathrm{ch}/\mathrm{d}\pT\mathrm{d}\eta }\xspace}
\newcommand {\Raa}       {\ensuremath{R_\mathrm{AA}}}
\newcommand {\RpPb}       {\ensuremath{R_\mathrm{pPb}}\xspace}
\newcommand {\Nevt}      {\ensuremath{N_\mathrm{evt}}}
\newcommand {\NevtINEL}  {\ensuremath{N_\mathrm{evt}(\textsc{inel})}}
\newcommand {\NevtNSD}   {\ensuremath{N_\mathrm{evt}(\textsc{nsd})}}
\newcommand{\dEdx}       {\ensuremath{\mathrm{d}E/\mathrm{d}x}\xspace}
\newcommand{\ttof}       {\ensuremath{t_\mathrm{TOF}}\xspace}
\newcommand {\ee}        {\mbox{$\mathrm {e^+e^-}$}\xspace}
\newcommand {\ep}        {\mbox{$\mathrm {e\kern-0.05em p}$}\xspace}
\newcommand {\pp}        {\mbox{$\mathrm {p\kern-0.05em p}$}\xspace}
\newcommand {\ppBoldMath} {\mbox{$\mathrm { \mathbf p\kern-0.05em \mathbf p }$}\xspace}
\newcommand {\ppbar}     {\mbox{$\mathrm {p\overline{p}}$}\xspace}
\newcommand {\PbPb}      {\ensuremath{\mbox{Pb--Pb}}\xspace}
\newcommand {\AuAu}      {\ensuremath{\mbox{Au--Au}}\xspace}
\newcommand {\CuCu}      {\ensuremath{\mbox{Cu--Cu}}\xspace}
\renewcommand {\AA}      {\ensuremath{\mbox{A--A}}\xspace}
\newcommand {\pA}        {\ensuremath{\mbox{p--A}}\xspace}
\newcommand {\pPb}       {\ensuremath{\mbox{p--Pb}}\xspace}
\newcommand {\Pbp}       {\ensuremath{\mbox{Pb--p}}\xspace}
\newcommand {\hPM}       {\ensuremath{h^{\pm}}\xspace}
\newcommand {\rphi}      {\ensuremath{(r,\phi)}\xspace}
\newcommand {\alphaS}    {\ensuremath{ \alpha_s}\xspace}
\newcommand {\MeanNpart} {\mbox{\ensuremath{< \kern-0.15em N_{part} \kern-0.15em >}}}

\newcommand {\sig}       {\ensuremath{S}\xspace}
\newcommand {\expsig}    {\ensuremath{\hat{S}}\xspace}
\newcommand {\prob}      {\ensuremath{P}\xspace}
\newcommand {\prior}     {\ensuremath{C}\xspace}
\newcommand {\prop}      {\ensuremath{F}\xspace}
\newcommand {\atrue}     {\ensuremath{\vec{A}_{\mathrm{true}}}\xspace}
\newcommand {\ameas}     {\ensuremath{\vec{A}_{\mathrm{meas}}}\xspace}
\newcommand {\detresp}   {\ensuremath{R}\xspace}

\newcommand {\pid}       {\ensuremath{\mathrm{\epsilon}_\mathrm{PID}}\xspace}
\newcommand {\nsigma}    {\ensuremath{\mathrm{n_{\sigma}}}\xspace}
\newcommand {\ylab} {\ensuremath{\mathrm{| y_{lab} |}}\xspace}

%
% -- | units | -------------------------------------------------------
%
\newcommand {\mass}     {\mbox{\rm MeV$\kern-0.15em /\kern-0.12em c^2$}}
\newcommand {\tev}      {\mbox{${\rm TeV}$}\xspace}
\newcommand {\gev}      {\mbox{${\rm GeV}$}\xspace}
\newcommand {\mev}      {\mbox{${\rm MeV}$}\xspace}
\newcommand {\kev}      {\mbox{${\rm keV}$}\xspace}
\newcommand {\tevBoldMath}  {\mbox{${\rm \mathbf{TeV}}$}}
\newcommand {\gevBoldMath}  {\mbox{${\rm \mathbf{GeV}}$}}
\newcommand {\mmom}     {\mbox{\rm MeV$\kern-0.15em /\kern-0.12em c$}}
\newcommand {\gmom}     {\mbox{\rm GeV$\kern-0.15em /\kern-0.12em c$}}
\newcommand {\mmass}    {\mbox{\rm MeV$\kern-0.15em /\kern-0.12em c^2$}}
\newcommand {\gmass}    {\mbox{\rm GeV$\kern-0.15em /\kern-0.12em c^2$}}
\newcommand {\nb}       {\mbox{\rm nb}}
\newcommand {\musec}    {\mbox{$\mu {\rm s}$}}
\newcommand {\nsec}     {\mbox{${\rm ns}$}}
\newcommand {\psec}     {\mbox{${\rm ps}$}}
\newcommand {\fmC}      {\mbox{${\rm fm/c}$}}
\newcommand {\fm}       {\mbox{${\rm fm}$}}
\newcommand {\cm}       {\mbox{${\rm cm}$}}
\newcommand {\mm}       {\mbox{${\rm mm}$}}
\newcommand {\mim}      {\mbox{$ \mu {\rm m}$}}
\newcommand {\cmq}      {\mbox{${\rm cm}^{2}$}}
\newcommand {\mmq}      {\mbox{${\rm mm}^{2}$}}
\newcommand {\dens}     {\mbox{${\rm g}/{\rm cm}^{3}$}}
\newcommand {\lum}      {\, \mbox{${\rm cm}^{-2} {\rm s}^{-1}$}}
\newcommand {\barn}     {\, \mbox{${\rm barn}$}}
\newcommand {\m}        {\, \mbox{${\rm m}$}}
\newcommand {\dg}       {\mbox{$\kern+0.1em ^\circ$}}
\newcommand{\mpp}{\ensuremath{\mathrm{pp}}\xspace}
\newcommand{\rts}{\ensuremath{\sqrt{s}}\xspace}
\newcommand{\GeV}{\ensuremath{\mathrm{GeV}}\xspace}
\newcommand{\TeV}{\ensuremath{\mathrm{TeV}}\xspace}
%\newcommand{\gevc}{\ensuremath{\mathrm{GeV}/c}\xspace}
\newcommand{\gevc}{GeV/\ensuremath{c}\xspace}
\newcommand{\GeVc}{\gevc}
\newcommand{\mevc}{\ensuremath{\mathrm{MeV}/c}\xspace}
\newcommand{\mevcc}{\ensuremath{\mathrm{MeV}/c^{2}}\xspace}
\newcommand{\gevcc}{\ensuremath{\mathrm{GeV}/c^{2}}\xspace}
\newcommand{\pt}{\ensuremath{p_{\rm T}}\xspace}
\newcommand{\kt}{\ensuremath{k_{\rm T}}\xspace}
\newcommand {\lumi}{\mathcal{L}_{\rm int}\xspace}
\newcommand{\nbinv}{\ensuremath{\rm nb^{-1}}}
\newcommand {\ubinv}{\ensuremath{\mu\rm b^{-1}}}
\newcommand {\um}{\ensuremath{\mu\rm m}\xspace}

\newcommand{\lt}{\textless}
\newcommand{\ctau}{\ensuremath{c\tau\xspace}}

%
% -- | some particles and decays | -------------------------------------------------------
%

\newcommand{\ePlusMinus}       {\mbox{$\mathrm {e^{\pm}}$}\xspace}
\newcommand{\muPlusMinus}      {\mbox{$\mathrm {\mu^{\pm}}$}\xspace}

\newcommand{\pion}            {\mbox{$\mathrm {\pi}$}\xspace}
\newcommand{\piZero}            {\mbox{$\mathrm {\pi^0}$}\xspace}
\newcommand{\piMinus}           {\ensuremath{\mathrm {\pi^-}}\xspace}
\newcommand{\piPlus}            {\ensuremath{\mathrm {\pi^+}}\xspace}
\newcommand{\piPlusMinus}       {\mbox{$\mathrm {\pi^{\pm}}$}\xspace}


\newcommand{\proton}    {\mbox{$\mathrm {p}$}\xspace}
\newcommand{\pbar}      {\mbox{$\mathrm {\overline{p}}$}\xspace}
\newcommand{\pOuPbar}   {\mbox{$\mathrm {p^{\pm}}$}\xspace}
\newcommand{\DZero}     {\mbox{$\mathrm {D^0}$}\xspace}
\newcommand{\DZerobar}  {\mbox{$\mathrm {\overline{D}^0}$}\xspace}
\newcommand{\Bminus}    {\mbox{$\mathrm {B^-}$}\xspace}
\newcommand{\BZero}     {\mbox{$\mathrm {B^0}$}\xspace}
\newcommand{\BZerobar}  {\mbox{$\mathrm {\overline{B}^0}$}\xspace}

\newcommand{\Dmes}       {\mbox{$\mathrm {D}$}\xspace}
\newcommand{\Lc}         {\mbox{$\mathrm {\Lambda_{c}}$}\xspace}
\newcommand{\Lb}{\ensuremath{\rm {\Lambda_b}}\xspace}
\newcommand{\Xic}         {\mbox{$\mathrm {\Xi_{c}}$}\xspace}
\newcommand{\lambdab}     {\mbox{$\mathrm {\Lambda_{b}^{0}}$}\xspace}
\newcommand{\lambdac}     {\mbox{$\mathrm {\Lambda_{c}^{+}}$}\xspace}
\newcommand{\xicz}        {\mbox{$\mathrm {\Xi_{c}^{0}}$}\xspace}
\newcommand{\xiczp}        {\mbox{$\mathrm {\Xi_{c}^{0,+}}$}\xspace}
\newcommand{\xicp}        {\mbox{$\mathrm {\Xi_{c}^{+}}$}\xspace}
\newcommand{\xib}        {\mbox{$\mathrm {\Xi_{b}}$}\xspace}
\newcommand{\LambdaParticle}        {\mbox{$\mathrm {\Lambda}$}\xspace}

\newcommand{\rmLambdaZ}         {\mbox{$\mathrm {\Lambda}$}\xspace}
\newcommand{\rmAlambdaZ}        {\mbox{$\mathrm {\overline{\Lambda}}$}\xspace}
\newcommand{\rmLambda}          {\mbox{$\mathrm {\Lambda}$}\xspace}
\newcommand{\rmAlambda}         {\mbox{$\mathrm {\overline{\Lambda}}$}\xspace}
\newcommand{\rmLambdas}         {\mbox{$\mathrm {\Lambda \kern-0.2em + \kern-0.2em \overline{\Lambda}}$}\xspace}

\newcommand{\Vzero}             {\mbox{$\mathrm {V^0}$}\xspace}
\newcommand{\Vzerob}             {\mbox{{\bold $\mathrm {V^0}$}}\xspace}
\newcommand{\Kzero}             {\mbox{$\mathrm {K^0}$}\xspace}
\newcommand{\Kzs}               {\ensuremath{\mathrm {K^0_S}}\xspace}
\newcommand{\phimes}            {\ensuremath{\mathrm {\phi}}\xspace}
\newcommand{\Kminus}            {\ensuremath{\mathrm {K^-}}\xspace}
\newcommand{\Kplus}             {\ensuremath{\mathrm {K^+}}\xspace}
\newcommand{\Kstar}             {\mbox{$\mathrm {K^*}$}\xspace}
\newcommand{\Kplusmin}          {\mbox{$\mathrm {K^{\pm}}$}\xspace}
\newcommand{\Jpsi}              {\ensuremath{\rm J/\psi}\xspace}
\newcommand{\DtoKpi}{\ensuremath{\rm D^0\to K^-\pi^+}\xspace}
\newcommand{\DtoKpipi}{\ensuremath{\rm D^+\to K^-\pi^+\pi^+}\xspace}
\newcommand{\DstartoDpi}{\ensuremath{\rm D^{*+}\to D^0\pi^+}\xspace}
\newcommand{\Dzero}{\ensuremath{\mathrm {D^0}}\xspace}
\newcommand{\Dzerobar}{\ensuremath{\mathrm{\overline{D}^0}}\xspace}
\newcommand{\Dstar}{\ensuremath{\rm D^{*+}}\xspace}
\newcommand{\Dplus}{\ensuremath{\rm D^+}\xspace}
\newcommand{\decleng}{\ensuremath{\rm L}_{xyz}}
\newcommand{\Lcminus}{\ensuremath{\rm {\overline{\Lambda}{}_c^-\xspace}}}
\newcommand{\Lcplus}{\ensuremath{\rm {\Lambda_c^+}\xspace}}
\newcommand{\Lbzero}{\ensuremath{\rm {\Lambda_b^0}}\xspace}
\newcommand{\LctopKpi}{\ensuremath{\rm \Lambda_{c}^{+}\to p K^-\pi^+}\xspace}
\newcommand{\LbtoLc}{\ensuremath{\rm \Lambda_{b}^{0}\to \Lc + \rm{X}}\xspace}
\newcommand{\LctopKzS}{\ensuremath{\rm \Lambda_{c}^{+}\to p K^{0}_{S}}\xspace}
\newcommand{\LctoenuLambda}{\ensuremath{\rm \Lambda_{c}^{+}\to e^{+} \nu_{e} \Lambda}\xspace}
\newcommand{\cosP}{\ensuremath{\rm cos_{\Theta_{pointing}}}\xspace}
\newcommand{\KzStopippim}{\ensuremath{\rm K^{0}_{S}\to \pi^{+} \pi^{-}}\xspace}
\newcommand{\Lambdatoppim}{\ensuremath{\rm \Lambda \to p \pi^{-}}\xspace}
\newcommand{\nue}{$\nu_e$}
\newcommand{\DtopiKzs}{\ensuremath{\rm D^+\to \pi^+ K^{0}_{S}}\xspace}
\newcommand{\DstoKKzs}{\ensuremath{\rm D_s^+\to K^+ K^{0}_{S}}\xspace}

\newcommand{\ptLc}{\ensuremath{p_{\rm T, \Lambda_c}}\xspace}
\newcommand{\ptpion}{\ensuremath{p_{\rm T, \pi}}\xspace}
\newcommand{\ptK}{\ensuremath{p_{\rm T, K}}\xspace}
\newcommand{\ptproton}{\ensuremath{p_{\rm T, \proton}}\xspace}
